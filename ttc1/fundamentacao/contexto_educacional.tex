% DEPRECATED!!!!

% \subsection{Redes Sociais no Contexto Educacional} \label{sec:networkseducation}

% A influência social e a disseminação de comportamento também causam profundos efeitos sobre os hábitos de estudantes, especialmente devido à grande quantidade de tempo que estes passam interagindo com seus colegas de classe, sendo esta uma característica observada principalmente no período da adolescência \cite{Butler-Barnes2015}. Na literatura científica, tem-se um considerável arcabouço de estudos que investigam padrões de características de propagação de influência e de que forma seus efeitos se manifestam nos estudantes, buscando correlações entre seu contexto social e outras variáveis de interesse \cite{Flashman2012,Gremmen2017,Farmer1996,Blansky2013,Rambaran2017,Berndt1990}.

% \citeonline{Goodenow1993} argumentam que a motivação acadêmica não pode ser analisada como um fenômeno exclusivamente individual, pois este é afetado profundamente pelas relações sociais e laços de amizade dos estudantes. As autoras também sugerem que a sensação de pertencimento à escola e a influência dos valores de amigos são fatores altamente importantes quando se trata de motivação acadêmica, sendo que este último apresenta um nível de correlação levemente inferior -- porém, no quesito da valorização das atividades acadêmicas, a influência das amizades apresentou correlação elevada. O estudo também identificou que o impacto do pertencimento é mais significativo em grupos com maior risco de evasão escolar.

% Reforçando a noção do impacto do contexto social de estudantes sobre seu comportamento escolar, \citeonline{Rice2013} identificaram que alunos relatam atitudes e percepções mais positivas, além de maior autoconfiança, acerca do estudo de matemática e ciências quando estes recebem suporte social abundante de seus seus pais, professores e amigos. O trabalho identificou que todos os três fatores apresentam correlações similares entre si (sendo todas positivas e estatisticamente significantes), porém o relato dos alunos amostrados sugere que o suporte social do grupo de amigos é consideravelmente inferior ao suporte oferecido por pais e professores. Sendo assim, sugere-se que a influência de laços de amizade merece um estudo mais aprofundado, por ser um fator destoante dos demais.