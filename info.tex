
\begin{Info}
% Universidade
{UNIVERSIDADE DO VALE DO ITAJAÍ}
% Escola
{ESCOLA DO MAR, CIÊNCIA E TECNOLOGIA}
% Curso
{CURSO DE CIÊNCIA DA COMPUTAÇÂO}
% Titulo
{Investigação Quantitativa Quanto ao Papel de Círculos Sociais no Desempenho Discente no Ensino Superior}
% Autor
{Fernando Concatto}
% Cidade e Data
{Itajaí (SC), fevereiro de 2019}
% Nome da Área de concentração
{Redes Complexas}
% Orientador(a)
{Alex Luciano Roesler Rese, MSc.}
% Coorientador(a) <Nome do Coorientador(a)>, <Titulação> %%%%%%%% Se não tiver coorientador deixe vazio
{Rafael de Santiago, Dr.}
\end{Info}

% Para TTC II
% \begin{Dedicatoria}
% Dedicatória
% \end{Dedicatoria}

% \begin{Agradecimentos}
% Dankeschon
% \end{Agradecimentos}

% \begin{Epigrafe}
% ``All men trust fully the illusion of life. But is this so wrong?\\
% A construction, a façade, and yet... A world full of warmth and resplendence.\\
% Young Hollow, are you intent on shattering the yoke, spoiling this wonderful falsehood?''\\
% %
% %"Men are props on the stage of life, and no matter how tender, how exquisite...\\
% %A lie will remain a lie.
% %Young Hollow, knowing this, do you still desire peace?"\\
% -- Aldia, Scholar of the First Sin
% \end{Epigrafe}

\begin{Resumo}
CONCATTO, Fernando. Investigação Quantitativa Quanto ao Papel de Círculos Sociais no Desempenho Discente no Ensino Superior. Itajaí, 2019. \pageref{LastPage} f. Trabalho Técnico-científico de Conclusão de Curso (Graduação em Ciência da Computação) -- Escola do Mar, Ciência e Tecnologia, Universidade do Vale do Itajaí, Itajaí, 2019.

O comportamento individual dos seres humanos é suscetível a influências advindas de seus contatos mais próximos. Sabe-se que o contato entre indivíduos, tanto direto quanto indireto, pode fomentar ou amenizar uma considerável diversidade de características e comportamentos humanos, como desenvolvimento de obesidade, dependência de álcool e drogas, opiniões positivas quanto a produtos e estados emocionais negativos. Este trabalho busca avaliar, de forma quantitativa, o impacto do contexto social de estudantes de graduação sobre seu desempenho acadêmico, compreendendo como contexto social o desempenho dos colegas de turma socialmente próximos a cada estudante. Tal investigação configura uma contribuição multidisciplinar, envolvendo áreas como Psicologia e Educação, concedendo uma percepção mais acurada acerca da importância da influência social no ambiente educacional. O estudo, que assumirá um caráter descritivo, envolverá a reconstrução das redes sociais presentes em cada turma de forma computacional, possibilitando a aplicação de técnicas de identificação de padrões entre indivíduos e seus pares. Para tanto, pretende-se elaborar e aplicar um questionário em turmas de cursos variados, correlacionando as redes reconstruídas com as tendências das médias das notas dos estudantes. Posteriormente, um conjunto de testes estatísticos será elaborado e aplicado para verificar: se o contexto social dos estudantes causa influência significativa em seu desempenho acadêmico; e se tal impacto varia de acordo com a área do conhecimento do curso que o aluno frequenta. Para identificar quais membros da turma compõem o contexto social do estudante, pretende-se aplicar a técnica conhecida na literatura científica como detecção de comunidades, que organiza a rede em subgrupos de acordo com a densidade de conexões de cada indivíduo. Evidências encontradas na literatura indicam a existência da influência descrita em turmas do ensino médio; espera-se, portanto, verificar se tal comportamento também se faz presente no ensino superior.

Palavras-chave: Redes Complexas. Detecção de Comunidades. Identificação de Padrões. Influência Social.
\end{Resumo}

\begin{Abstract}
The individual behavior of human beings is susceptible to influences coming from their close contacts. It is known that contact between individuals, both direct and indirect, can foster or ameliorate a considerable diversity of human characteristics and behaviors, such as the development of obesity, dependence on alcohol and drugs, positive opinions about products and negative emotional states. This work aims to quantitatively evaluate the impact of the social context of undergraduate students on their academic performance, understanding as social context the performance of classmates socially close to each student. This research constitutes a multidisciplinary contribution, involving areas such as Psychology and Education, giving a more accurate perception about the importance of social influence in the educational environment. The study, which will take a descriptive character, will involve the reconstruction of the social networks present in each class in a computational way, allowing the application of pattern recognition techniques among individuals and their peers. Therefore, we intend to elaborate and apply a questionnaire in classes of varied courses, correlating the reconstructed networks with the trends of the means of the students' grades. Subsequently, a set of statistical tests will be developed and applied to verify: if the social context of students causes significant influence on their academic performance; and if such impact varies according to the field of knowledge of the course that the student attends. To identify which class members make up the student's social context, we intend to apply the technique known in the scientific literature as community detection, which organizes the network into subgroups according to the density of connections of each individual. Evidence found in the literature indicates the influence described in high school classes; it is hoped, therefore, to verify if such behavior is also present in higher education.
\end{Abstract} 

